\subsection{Introducci� a la plataforma}
Android es una pila de programari per a dispositius m�bils que inclou un sistema operatiu basat en Linux, middwlware i aplicacions b�siques de les que disposen tots els tel�fons m�bils habituals.

Neix el 5 de novembre del 2007 i darrera seu tamb� neix la Open Handset Alliance, que engloba tot tipus de companyies com operadores telef�niques (Telef�nica, T-Mobile, Telekom Italia...), fabricants de dispositius m�bils (HTC, Samsung, Toshiba...), companyies de semiconductors, de Software (Google, eBay...) y altres consultores que aporten suport econ�mic y t�cnic a la plataforma.

El 9 de desembre del 2008  es va ampliar alian�a amb 14 nou membres, AKM Semiconductor Inc., ARM, ASUSTek Computer Inc., Atheros Communications, Borqs, Ericsson, Garmin International Inc., Huawei Technologies, Omron Software Co. Ltd, Softbank Mobile Corporation, Sony Ericsson, Teleca AB, Toshiba Corporation  i Vodafone. Aquest fet es important, ja que aix� fa que la possibilitat d'�xit d'aquesta plataforma lliure sigui real. Per posar un exemple, podem veure que ara mateix hi ha mes d'una operadora telef�nica amb actuaci� al estat espanyol, mentre que abans nom�s hi havia la opci� de Movistar. Un altre punt a remarcar �s que tots els fabricants de m�bils s'han adherit a la alian�a amb el comprom�s d'utilitzar la plataforma en els seus dispositius(no exclusivament), nom�s quedaria fora Apple i Nokia que cadasc� aposta pel seu propi sistema.

Un dels punts forts de la plataforma android es la pol�tica de considerar totes les aplicaci�ns iguals, incloses les de trucar, enviament de missatges curts de text, etc. Per aquest motiu qualsevol usuari pot canviar les aplicaci�ns b�siques del seu aparell modificant-lo al seu gust.

�nicament s'han reservat la opci�, suposadament per seguretat del usuari, de remotament i sense perm�s previ de l'usuari, desinstal�lar les aplicaci�ns que puguin considerar malicioses.

Todas las aplicaciones son iguales
Android no diferencia entre las aplicaciones de terceros y las aplicaciones b�sicas del tel�fono: incluso se puede sustituir la pantalla de inicio o el panel de marcaci�n.
