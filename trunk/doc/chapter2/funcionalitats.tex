\subsection{Funcionalitats}
Las aplicaciones de Android pueden acceder a las funciones principales de dispositivos m�viles mediante distintas API est�ndar. A trav�s de intenciones, las aplicaciones pueden anunciar sus funciones para que las utilicen otras aplicaciones.

Es f�cil insertar HTML, JavaScript y hojas de estilo en las aplicaciones. Una aplicaci�n puede representar contenido web a trav�s de WebView

Cualquier aplicaci�n de un dispositivo m�vil se puede sustituir o ampliar, incluso los componentes b�sicos como el panel de marcaci�n o la pantalla de inicio

Android es un completo entorno multitarea donde las aplicaciones se pueden ejecutar en paralelo. Mientras se ejecuta en segundo plano, una aplicaci�n puede producir notificaciones para obtener la atenci�n del usuario.

Applications

Android will ship with a set of core applications including an email client, SMS program, calendar, maps, browser, contacts, and others. All applications are written using the Java programming language.
Application Framework

Developers have full access to the same framework APIs used by the core applications. The application architecture is designed to simplify the reuse of components; any application can publish its capabilities and any other application may then make use of those capabilities (subject to security constraints enforced by the framework). This same mechanism allows components to be replaced by the user.

Underlying all applications is a set of services and systems, including:

    * A rich and extensible set of Views that can be used to build an application, including lists, grids, text boxes, buttons, and even an embeddable web browser
    * Content Providers that enable applications to access data from other applications (such as Contacts), or to share their own data
    * A Resource Manager, providing access to non-code resources such as localized strings, graphics, and layout files
    * A Notification Manager that enables all applications to display custom alerts in the status bar
    * An Activity Manager that manages the life cycle of applications and provides a common navigation backstack

For more details and a walkthrough of an application, see Writing an Android Application.
Libraries

Android includes a set of C/C++ libraries used by various components of the Android system. These capabilities are exposed to developers through the Android application framework. Some of the core libraries are listed below:

    * System C library - a BSD-derived implementation of the standard C system library (libc), tuned for embedded Linux-based devices
    * Media Libraries - based on PacketVideo's OpenCORE; the libraries support playback and recording of many popular audio and video formats, as well as static image files, including MPEG4, H.264, MP3, AAC, AMR, JPG, and PNG
    * Surface Manager - manages access to the display subsystem and seamlessly composites 2D and 3D graphic layers from multiple applications
    * LibWebCore - a modern web browser engine which powers both the Android browser and an embeddable web view
    * SGL - the underlying 2D graphics engine
    * 3D libraries - an implementation based on OpenGL ES 1.0 APIs; the libraries use either hardware 3D acceleration (where available) or the included, highly optimized 3D software rasterizer
    * FreeType - bitmap and vector font rendering
    * SQLite - a powerful and lightweight relational database engine available to all applications

Android Runtime

Android includes a set of core libraries that provides most of the functionality available in the core libraries of the Java programming language.

Every Android application runs in its own process, with its own instance of the Dalvik virtual machine. Dalvik has been written so that a device can run multiple VMs efficiently. The Dalvik VM executes files in the Dalvik Executable (.dex) format which is optimized for minimal memory footprint. The VM is register-based, and runs classes compiled by a Java language compiler that have been transformed into the .dex format by the included "dx" tool.

The Dalvik VM relies on the Linux kernel for underlying functionality such as threading and low-level memory management.
Linux Kernel

Android relies on Linux version 2.6 for core system services such as security, memory management, process management, network stack, and driver model. The kernel also acts as an abstraction layer between the hardware and the rest of the software stack.