\subsection{Punts d'estudi}
De totes les funcionalitats de les que presumeix Android, he decidit explorar a fons un bon n�mero d'elles, sobretot les relacionades amb el m�n multim�dia i internet.

\subsubsection{Reproducci� de fitxers MP3}
Explorar i verificar el funcionament de la reproducci� de fitxers MP3 sobre android. Dintre d'aquest punt es voldria explorar a conciencia els seg�ents punts:

-Accions habituals sobre un fitxer d'MP3, com seria reproduir, pausar, parar, passar can��, moure's en la linea de temps de la can��.

-Lectura dels meta tags dels MP3. Tots els MP3 tenen la possibilitat d'emagatzemar dades sobre la can��, com per exemple el t�tol, l'artista, l'album, la portadada del disc, etc... tot aix� mitjan�ant el format estandard ID3.

\subsubsection{Funcionament d'aplicaci�ns en seg�n pla}
Android �s un entorn multitasca, permet que les aplicaci�ns o fils treballin en paral�lel.
Mentre corren altres en primer pl�, les aplicaci�ns en seg�n pl� poden crear notificaci�ns. Aquestes notificaci�ns ser�n mostrades al usuari, per informarli d'algun event desitjat o solicitar la seva atenci� per partic�par amb la aplicaci� en segon pl�.

\subsubsection{Emmagatzematge de dades amb SQLite}
Android dona la possiblitat de emmagatzemar dades sobre SQLite i interactuar amb aquesta petita base de dades relacional igual que en altres entorns com pot ser PCs.

\subsubsection{Streaming d'audio}
Android anunc�a la possiblitat de poder reproduir streamings 