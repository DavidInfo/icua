\subsection{Descripci� detallada de l'aplicaci�}
Un cop estudiada la documentaci� existent de la plataforma de desenvolupament d'android, per poder assolir la finalitat del projecte, realitzarem una aplicaci� real per a Android. S'ha buscat un perfil d'aplicaci� adient per investigar el m�xim de les possibilitats que ens ofereix la plataforma. \newline\newline
Aquesta aplicaci� seria un reproductor de m�sica per a m�bil, que ens hauria de permetre compartir la m�sica amb un amic o altra gent, a m�s, ha de permetre interactuar, de manera totalment transparent per a l'usuari, amb aplicacions web, ja sigui nom�s amb una o moltes simult�niament (Mashups \cite{wiki:mashup}).\newline\newline
El desenvolupament de l'aplicaci� ha de ser sobre la versi� de SDK de cada moment i finalment ha de ser compatible amb la release oficial d'Android, fent possible la seva utilitzaci� sobre un terminal real.
Podr�em tenir en compte els seg�ents m�duls per diferenciar parts independents.\newline \newline
M�dul 1-  Reproductor MP3 :\newline
Aquesta �s la part m�s gruixuda de l'aplicaci�, ja que inclou la reproducci� de mp3, enregistrament i gesti� d'autors, �lbums i fotos de cada can��, interficie gr�fica de la base del programa.\newline\newline
M�dul 2 - Streaming i compartici� de m�sica :\newline
Aquest m�dul es dedicaria a l'streaming de la can�� que est� sonant i a la compartici� de la m�sica amb altres usuaris.\newline\newline
M�dul 3 - Mashups i altres funcionalitats :\newline
Aquest altre m�dul es podria considerar com un conjunt de petits m�duls amb plug-ins que es considerin interessants per destacar possibilitats de la plataforma, com per exemple interacci� amb APIs com LastFM.\newline
