\subsection{ \LaTeX }

Per a la escriptura de la mem�ria, he escollit \LaTeX. \LaTeX es un llenguatge de marques molt conegut en el ambient t�cnic i amb llic�ncia LPPL.

\subsubsection{Per qu� \LaTeX ?}

Hi han hagut diferents motius que m'han incentivat a escollir \LaTeX.

-La inicial, la proposta del meu tutor(Toni Riera) per la seva utilitzaci� com a medi per escriure la mem�ria del projecte.

-Un altre motiu per escollirl-o �s la facilitat que dona \LaTeX independitzar contingut del format. Pots dedicar-te �nicament a escriure i despr�s aplicar el format que desitjis a tot el document.

-Un altre aspecte molt important per a una mem�ria es el fet de poder dividir el contingut en fitxers diferents, aquest fet de revisar fitxers amb menys contingut fa que tota la mem�ria sigui molt mes f�cil de moure i reordenar contingut.

-\LaTeX tamb� es molt �til per la seva gesti� autom�tica de la separaci� de sil�laves a l'hora de tallar les l�nies.

-Un altre avantatge de \LaTeX sobre altres, es la molt bona gesti� de les p�gines, cap�tols quan decideixes realitzar un document estil llibre com �s aquesta mem�ria. Ell sol et comen�a els cap�tols a la plana de la dreta, arregla es marges segons sigui p�gina parell o imparell,i altres detalls que faciliten molt la feina.

-La gesti� de Bilbilografia amb BibteX �s tamb� una de les claus per aquesta elecci�, senzill d'utilitzar i amb resultats francament bons.

-I finalment, per� no la menys important, el fet d'anteriorment nomes haver utilitzat \LaTeX  un cop. Considero aquesta una oportunitat molt bona de familiaritzarme molt m�s amb \LaTeX, per aix� poder aplicar els coneixements a qualsevol altre document que pugui escriure en un futur.

\subsubsection{Utilitzaci� de \LaTeX}

Per poder utilitzar \LaTeX al meu entorn,(Una debian stable amb Gnome) he decidit utilitzar Gedit, que es el editor que ja porta per defecte Gnome.
Nom�s he hagut d'instal�lar el plugin de \LaTeX per a Gedit i el compliador de \LaTeX rubber, aquest s'encarrega de la seva conversi� a pdf.
